\chapter{A ferramenta desenvolvida}

	Para o desenvolvimento do software livraria Dom Casmurro foi utilizado as seguintes ferramentas:


\section{Linguagem de programação}
 A linguagem de programação utilizada foi a linguem Java, muito utilizada em desenvolvimento para desktop e web. É uma linguagem orientada a objetos,fortemente tipada e multiplataforma, de modo que uma vez compilado o código, é possivel executa-lo em qualquer sistema que disponha de uma maquina virtual Java;

\section{Ambiente de desenvolvimento}
 O ambiente utilizado para o desenvolvimento do código do programa foi o software NetBeans IDE versão 8.0.1; focada no desenvolvimento utilizando a linguagem de programação Java.

\section{Persistência de dados}
 O armazenamento das informações como por exemplo os cadastro das operações de cliente, funcionários, fornecedores e livros são inseridos no banco de dados Mysql Server 5.6;

O software utilizado para o desenvolvimento do banco de dados foi o MySQL Workbench 6.0 CE. Tem como função a modelagem e gerenciamento dos dados, onde gera as tabelas e seus relacionamentos, podendo inserir dados nessas tabelas e efetuar a sincronização entre modelo lógico e a base de dados física.


\section{Requisitos Funcionais}

\begin{itemize}

\item Ao abrir o software a tela de login é exibida, mostrando os campos “usuário” e “senha”, com dois botões, um “acessar” o software, e outro para “cancelar” a operação;

\item Se o botão “Cancelar” for pressionado é exibida uma mensagem de aviso: “Tem certeza que deseja encerrar a operação?” se “sim” a mensagem de aviso some e o programa é automaticamente fechado; se “não” a mensagem é fechada e permanece na tela de login;

\item Se o botão “Acessar” for pressionado o software verifica se a senha está correta: “empresa123”, a tela principal do software é exibida. E a tela de login é fechada. Caso a “senha” esteja incorreta o software exibe uma mensagem de aviso: “Senha incorreta” e limpa os campo para que possa ser novamente preenchido.

\item Na tela principal do software é exibido um menu com as seguintes informações “Cadastrar”, “Consultar”, “Editar”, “Deletar” e “sobre”;

\item Se qualquer uma das opções do menu for pressionado: será exibido um submenu: “Cliente”, “Funcionário”, “Fornecedor”, “Livros”;

\item No “cadastro” o usuário poderá cadastrar os “Cliente”, “Funcionário”, “Fornecedor”, “Livros”, na base de dados;

\item Na “Consulta” o usuário poderá ter acesso as informações armazenada na base de dados, de qualquer uma das opções mencionadas acima;

\item Em “Editar” o usuário poderá acrescentar, remover ou alterar informações dos cadastros feitos;

\item Caso seja necessário o menu com a opção “Deletar” fará com que o usuário possa remover os  cadastros realizados até o então.

\item Se  “Sobre” do menu principal for selecionado será exibido uma tela com as informações do software e quem o desenvolveu;


\end{itemize}


\section{Requisitos Não Funcionais}

\begin{itemize}

\item O software é deve ser desenvolvido na linguagem de programação Java, aproveitando sua natureza mutiplataforma e pela mesma fazer parte da ementa do curso Técnico em Informática;

\item Para executar nosso programa com sobra de recursos é necessário uma máquina com um processador Intel Atom de 1,7 GHZ com 2GB de memória RAM;

\item É executado em qualquer plataforma seja ela Linux, Windows;
\end{itemize}

\section {UML do software (Diagrama de USE CASE)}

	A construção da UML no desenvolvimento do software trouxe vários benefícios, pois nos auxiliou na modelagem e documentação do sistema. Nele foi construído as definições de cada uma das operações;


\begin{figure}
	\centering 
	\caption{UML Cliente}
	
	\label{uml_cliente}
	\includegraphics[scale = 0.45]{imagens/uml-cliente.jpg}
	\\Fonte: Do autor
\end{figure}