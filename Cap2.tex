\chapter{A empresa em questão}

\section{A empresa}

	Livraria Dom Casmurro, é uma empresa simulada desenvolvida pela turma de Administração da instituição Senac, criada com o intuito de proporcionar a comercialização de livros novos e semi novos.
	Para entendermos um pouco mais sobre a empresa Livraria Dom Casmurro temos o seguinte depoimento de um dos integrantes da equipe:

\begin{quote}
	"A nossa empresa começou a partir de uma pesquisa mercadológica realizada na instituição SENAC de Itajubá, na qual nossa equipe, formada por sete administradores, tinha em mãos quatro ideias de negócios, dentre elas a escolhida foi à livraria. 

Após essa etapa, a ideia foi moldada e a empresa foi sendo construída junto ao nome, que depois de algumas sugestões, foi escolhido o nome Livraria Dom Casmurro. Foi escolhido o nome Dom Casmurro pelo fato de ser um clássico literário, escrito por Machado de Assis, do ano de 1899, que remete a uma empresa clássica e com tradição. 

Desde então nossa livraria vem trabalhando e prezando sempre a sustentabilidade. Nossa loja física que se localiza na rua Dr. Américo de Oliveira foi planejada com matérias sustentáveis, e reutilizamos a energia solar e a água da chuva. 
Nos dá Dom Casmurro estamos oferecendo aos nossos clientes os melhores produtos, os melhores preços tanto na loja física quanto na loja virtual, mostrando que a Livraria Dom Casmurro tem muito potencial para ser a melhor livraria de Itajubá e região em um futuro bem próximo."


\raggedleft
Depoimento realizado por: Cléverson William – Turma de Administração
\end{quote}



\section{Necessidade da criação do software para a empresa}

	No desenvolvimento do projeto, teve-se como função informatizar a livraria pra que o processo das vendas e informações dos clientes, fossem cadastrados de forma rápida, e segura.
	
	Antes de passar por esse processo, a empresa recorria aos papeis para fazer os cadastros dos clientes, e promissórias feitas a mão para as vendas dos produtos. Este método de armazenamento dificultava muito a organização da empresa, pois eram custosos e inseguros, onde os  expunham a riscos como perda das informações, e degradação dos arquivos.
	
	Considerando os pontos acima destacados, com os conhecimentos adquiridos em sala de aula, desenvolveu-se um projeto de software que informatizasse as tarefas executadas diariamente na empresa como as de cadastros, consultas, edições e remoções da base de dados. Proporcionando um armazenamento seguro, e rápido;

\section{Problemas enfrentados antes da criação do software}


	Uma das dificuldades encontradas no processo de desenvolvimento do software foi em relação aos levantamentos dos requisitos. Houve um processo para entender o que a empresa gostaria que fosse desenvolvidos a eles, pois muitas das ideias expostas pela empresa não eram tão especificas, e contundentes com as suas necessidades encontradas nas conversas com as mesmas. 
	O conhecimento especifico foi mais um ponto relevante neste quesito, antes de ter construído esse software não havíamos projetado um programa tão grande, com uma plataforma com tantas funções;

\section{Problemas que o software irá suprir}


\begin {itemize}
\item Rapidez no cadastro, consulta, edição, e remoção das operações registradas;
\item Armazenamento das informações de forma mais segura;

\end{itemize}





