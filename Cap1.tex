\chapter{Introdução}

\section {Proposta do trabalho}

	A proposta inicial do projeto consisti em desenvolver um software de gerência de produtos para a empresa simulada, livraria Dom Casmurro. A livraria, até o momento, gerenciava seus cadastros de produtos e clientes de maneira totalmente manual, sem auxilio de um sistema informatizado.

	A partir de conversas com a empresa, foi realizado o levantamento do requisitos para o desenvolvimento do software, encontrando quais eram os pontos pelos quais poderíamos  suprir e resolver as deficiências apresentadas pelo antigo sistema de gestão da livraria. Entre os requisitos a serem satisfeitos, destaca-se a necessidade da realização das operações de cadastros e consultas de livros e clientes de forma rápida e segura. 

	Com base nos pontos apresentados acima, foi implementado um software que oferece todos os processos que a empresa executaria internamente: operação de adição, edição, consulta e remoção de cadastros de clientes, funcionários, fornecedores e livros.

	Outro ponto a ser destacado, é a preocupação quanto a segurança no armazenamento dos dados da empresa, necessidade que foi suprida através da utilização de ferramentas de persistência de dados que serão detalhadas posteriormente.
	
\section{Justificativa}

	A empresa contratante, livraria Dom Casmurro, até o momento do início do projeto, executava todas as suas operações de maneira totalmente manual, sendo que os dados de seus clientes, empregados, fornecedores e produtos eram registrados em papel, sendo esse um método de armazenamento de informações custoso e inseguro, uma vez que demanda espaço para armazenamento dos documentos, os expõe ao risco de degradação através do ambiente, por meio da umidade e poeira, além de também facilitar o acesso de informação por pessoas não autorizadas. 

Considerando os pontos acima destacados, desenvolveu-se um projeto de software que informatizasse as tarefas de cadastro, consulta, de clientes, funcionários, fornecedores e livros que aumentaria a rapidez que essas operações seriam realizadas, permitindo que o funcionários da livraria atendessem uma demanda maior de clientes em menos tempos, obtendo assim mais lucro. Ao realizar o armazenamento dos dados de um sistema de gerenciamento de banco de dados, permite-se o acesso aos dados somente por pessoas autorizadas e garante alta disponibilidade dos dados, através do uso de backups. Além disso, o uso do banco de dados também elimina a necessidade do cliente de possuir cópias impressas de documentos, reduzindo gastos com impressão do papel.


\section{Objetivo}
	

\subsection {Trazer mais segurança para os dados da empresa}
Assegurar ao cliente que todos as operações realizadas internamente pela empresa, sendo elas de cadastro, edição, remoção ou consulta na base de dados não sejam  perdidas ao decorrer do processo, assegurando que somente pessoas autorizadas obtenham acesso a essas informações;

\subsection{Ter praticidade ao efetuar as operações necessárias}
Ser um software que pressa pela usualidade das operações realizadas pelos usuários. Tornando-o simples seu manuseio, fazendo com que pessoas leigas na área da informática consiga efetuar as operações de forma eficaz; 

\subsection{Permitir o controle da base de dados}
Permitir que o funcionário tenha acesso direto aos dados efetuados internamente na empresa, e possa realizar edições e remoções caso necessário na base de dados;

\subsection{Fornecer um software com estabilidade}
Permitir que o software com o java instalado seja executado em qualquer sistema operacional sendo eles: Linux, Windows ou Mac;
	

\section{Escopo do Projeto}

O escopo do projeto foi criado a partir do desenvolvimento em sala de aula;	

\begin{center}
	\textbf{COLOCAR TABELA}
\end{center}


Descrição do escopo do projeto:

\begin {enumerate}

\item Entrevista com a empresa- Foi avaliado o funcionamento da mesma, e extrairmos as ideias de como um software poderia auxiliar nas operações realizadas internamente;

\item Levantamento das ideias para o desenvolvimento do software – Após a avaliação com o cliente obteve-se varias ideias, algumas delas foram desenvolvidas, já outras foram desfeitas, pelo fato de não ser tão primordial ao processo de desenvolvimento do software, onde será citado a diante;

\item Conversa com a empresa - Conclusão das ideias obtidas-  Logo após expor as ideias ao cliente, teve-se como retorno a aprovação do software;


\item Levantamento dos requisitos funcionais e não funcionais  - Realizou-se os levantamentos dos requisitos;

\item Criação dos casos de usos  - Desenvolveu-se o caso de uso a partir das requisições levantadas;

\item Criação do layout do programa – Com os casos de usos e os levantamentos dos requisitos  concluídos, desenvolveu-se o layout do programa, preocupando-se com a usualidade das operações realizadas pelos usuários;

\item Desenvolvimento do banco de dados em Mysql – Ponto importante na criação do software, aqui serão armazenados todas as informações de cadastros;

\item Criação do software em Java – Desenvolvimento do projeto;


\item Teste do software – Realiza-se as precauções necessárias, e são consertados bugs no sistema;

\item Criação da documentação do projeto – desenvolvimento do relatório;

\end{enumerate}